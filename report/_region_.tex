\message{ !name(report.tex)}\documentclass{article}
\usepackage{graphicx}

\title{Approaches to Scamming PageRank}
\author{Manu Goyal, Don-Wook Shin, Luvsanbyamba Buyankhuu}
\date{\today}

\begin{document}

\message{ !name(report.tex) !offset(-3) }


\maketitle

\section*{Introduction}

The PageRank algorithm we studied in class is an effective way to sort pages by
their relative importance. Unfortunately, there are numerous ways malicious
users can modify the network in order to artificially increase the PageRank
score of desired pages. We explore different exploitations of PageRank in the
context of the Simple English Wikipedia network.

\section*{Methods}

We chose to use the Simple English Wikipedia network of pages, because it was a
somewhat large, real-world dataset, but wasn't so large that experimenting with
the data would become too cumbersome. The first thing we did was to calculate
PageRank on the entire wikipedia network, using the standard PageRank algorithm
(which we found on Wikipedia). After looking over the results we got, we devised
two ways to scam the PageRank algorithm, to boost the rank of a specific page.

\subsection*{Method 1: Create Links From Important Pages}

This method is fairly straightforward. Since the rank of a page improves when
other pages with a high rank link to it, somebody could easily improve the rank
of a given Wikipedia page by linking other important pages to it.

To see exactly how much this would improve the score, let $\pi_L$ be the rank of
a low-ranked page, and $\pi_H$ be the rank of a high-ranked page that we will
add a link to. Initially, after PageRank has converged $\pi_L$ will satisfy the
equation $\pi_L = (1 - d) + d(\sum_{i=0}^n \frac{\pi_i}{L(i)}$, where $d$ is a
damping factor for successive iterations of PageRank, and $L(i)$ is the number
of outbound links for page $i$.


\end{document}

\message{ !name(report.tex) !offset(-48) }
